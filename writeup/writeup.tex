\documentclass[twocolumn,11pt]{article}
\usepackage{authblk}
\usepackage{graphicx}

\title{Tor Inter-Relay Latency}
\author[1]{Frank Cangialosi}
\author[1]{Scott DellaTorre}
\author[1]{Emily Kowalczyk}
\author[1]{Brent Schlotfeldt}
\author[1]{Dave Levin\thanks{Faculty Advisor}}
\affil[1]{University of Maryland}
\date{}
\raggedbottom

\begin{document}

\maketitle

\section {Introduction}

Network latency tools such as ping and traceroute calculate the round trip time of sending small data packets from an originating host to a destination host.

These tools, however, can only use the local machine as the originating host, and  therefore can only generate measurements that originate from that machine.

But suppose you are able to measure the latency between two arbitrary hosts that you do not control.  For example, what if you could tell a random computer A to ping some other random computer B?  And what if you could do this for many computers in many countries many times?  Could you craft a map of all the latencies between hosts in a network? How could you use this information to benefit the network?

Our research took these questions and applied them to the unique structure of the Tor network.  Tor is a popular anonymity-preserving network, consisting of routers run by volunteers all around the world.  It protects a user’s privacy by relaying their network traffic through a series of routers, called a Tor circuit.  Tor circuits typically are generated by a Tor entry node but also can be constructed manually by the client.

Our approach was to construct specific sets of Tor circuits such that with some simple set manipulations we could find the latent measurements described above. More precisely, we constructed three circuits A, B, C where A contained nodes [W, X, Y, Z], B contained nodes [W, X] and C contained [Y, Z].  By calculating the round trip times of each, we could then subtract the round trip time of A – round trip time of B – round trip time of C to find the latency of [X, Y]- two unknown, intermediary relays.  We took such measurements over a sample of \_ Tor relays and discuss the results and implications such as \_.

The rest of our paper progresses as follows.  In section 2, we discuss related work on measuring latency and Tor’s performance.  In section 3, we describe our study and how our times were gathered.  Section 4 presents and analyzes our experiment’s results, and Section 5 concludes and discusses possible future work.

\section{Related Work}

We found a tool that computes the round-trip times for entire Tor circuits \cite{rtt}.

\section{Procedure}

\section{Results}

\section{Conclusion}

%\begin{figure}[th]
%\includegraphics[width=3in]{graph}
%\caption{\label{fig:RTTs}}
%\end{figure}

%\subsection{Proof of Stuff}

%\begin{equation}
%	x - y = -4
%\end{equation}

\bibliography{biblio}
\bibliographystyle{abbrv}

\end{document}
